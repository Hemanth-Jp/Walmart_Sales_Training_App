%%%%%%%%%%%%%%%%%%%%%%%%
%
% $Autor: Hemanth Jadiswami Prabhakaran $
% $Datum: 2025-06-30 11:11:43Z $
% $Pfad: GitHub/BA25-01-Time-Series/report/Contents/en/kddDataTransformationAndMining.tex $
% $Version: 1 $
%
% $Project: BA25-Time-Series $
%
%%%%%%%%%%%%%%%%%%%%%%%%

\chapter{KDD Data Transformation and Mining}

\section{Data Transformation}

The Walmart sales forecasting project implements comprehensive data transformation processes that convert raw CSV files into analysis-ready time series data suitable for machine learning algorithms. The transformation pipeline follows KDD best practices while addressing specific challenges inherent in retail sales data including temporal alignment, missing value handling, and feature engineering.

\textbf{Multi-Source Data Integration}: The transformation process begins with integrating three distinct data sources: sales history (train.csv), economic indicators (features.csv), and store characteristics (stores.csv). This integration requires careful temporal alignment and key matching to ensure data consistency across different granularities.

\begin{lstlisting}[style=bashstyle, caption={Data Integration and Preprocessing Pipeline}]
# Load and merge multi-source data
df = load_and_merge_data(train_file, features_file, stores_file)
df = clean_data(df)
df_week, df_week_diff = prepare_time_series_data(df)
\end{lstlisting}

\textbf{Temporal Data Preparation}: The transformation process aggregates daily sales data to weekly granularity and applies differencing to achieve stationarity required for time series modeling. This includes date parsing, temporal sorting, and creation of lag variables that capture autocorrelation patterns essential for forecasting accuracy.

\textbf{Feature Engineering}: Advanced feature engineering creates derived variables including holiday indicators, seasonal decomposition components, and external economic variable integration. The transformation handles missing values through interpolation methods that preserve temporal continuity while maintaining data quality.

\textbf{Data Quality Enhancement}: The pipeline implements robust data cleaning procedures including outlier detection that distinguishes between legitimate extreme events (holiday sales spikes) and data quality issues, ensuring model training on high-quality, representative data.

\section{Data Mining Application to Walmart Sales Forecasting}

The Walmart sales forecasting application exemplifies practical data mining implementation through systematic application of KDD principles to real-world retail analytics challenges. The project addresses over 4,400 individual time series across multiple stores and departments, requiring scalable data mining approaches.

\textbf{Domain-Specific Pattern Discovery}: The data mining process identifies critical retail patterns including weekly seasonality, annual cycles, and holiday effects that significantly impact sales performance. Pattern discovery incorporates both univariate time series analysis and multivariate relationships between sales and economic indicators.

\textbf{Hierarchical Data Mining}: The application handles hierarchical data structures across stores and departments, implementing data mining techniques that can capture both individual time series characteristics and cross-sectional patterns that emerge across similar retail locations.

\textbf{Scalable Algorithm Implementation}: The data mining framework employs efficient algorithms capable of processing thousands of time series while maintaining computational performance. This includes automated hyperparameter optimization and parallel processing capabilities essential for large-scale retail analytics.

\textbf{Business-Relevant Knowledge Extraction}: The data mining process extracts actionable insights including demand patterns, seasonal trends, and economic sensitivity measures that directly support inventory management, staffing decisions, and strategic planning initiatives.

\section{Hyperparameters}

The Walmart forecasting system implements comprehensive hyperparameter management for both traditional statistical methods (ARIMA) and modern smoothing techniques (Exponential Smoothing), balancing automated optimization with user customization capabilities.

\subsection{Auto ARIMA Hyperparameters}

\textbf{Parameter Space Configuration}: Auto ARIMA employs intelligent parameter space exploration with configurable bounds that prevent overfitting while ensuring comprehensive model evaluation.

\begin{lstlisting}[style=bashstyle, caption={ARIMA Hyperparameter Configuration}]
CONFIG = {
    'DEFAULT_MAX_P': 20,           # Maximum AR terms
    'DEFAULT_MAX_Q': 20,           # Maximum MA terms  
    'DEFAULT_MAX_P_SEASONAL': 20,  # Maximum seasonal AR terms
    'DEFAULT_MAX_Q_SEASONAL': 20,  # Maximum seasonal MA terms
}
\end{lstlisting}

\textbf{Automated Parameter Selection}: The system employs grid search optimization with AIC/BIC criteria to automatically determine optimal (p, d, q) and seasonal (P, D, Q, s) parameters, eliminating subjective model specification while ensuring statistical rigor.

\textbf{Stationarity Assessment}: Hyperparameter optimization includes automated stationarity testing through augmented Dickey-Fuller tests, automatically determining differencing requirements without manual intervention.

\subsection{Exponential Smoothing Hyperparameters}

\textbf{Seasonal Component Configuration}: Holt-Winters hyperparameters control seasonal pattern modeling through additive and multiplicative formulations optimized for retail data characteristics.

\begin{lstlisting}[style=bashstyle, caption={Exponential Smoothing Hyperparameter Setup}]
hyperparams = {
    'seasonal_periods': 20,      # Weekly retail cycles
    'seasonal': 'additive',      # Seasonal component type
    'trend': 'additive',         # Trend component type
    'damped': True              # Prevent over-extrapolation
}
\end{lstlisting}

\textbf{Adaptive Smoothing Parameters}: The optimization process automatically determines optimal smoothing constants for level, trend, and seasonal components, enabling adaptive response to changing data patterns while maintaining forecasting stability.

\section{Input Data Processing}

The input processing pipeline transforms raw CSV data into structured time series suitable for machine learning algorithms, implementing comprehensive validation and preprocessing procedures.

\textbf{File Upload and Validation}: The system accepts three required CSV files (train.csv, features.csv, stores.csv) with comprehensive validation including format verification, column presence checking, and data type validation to ensure processing compatibility.

\textbf{Data Integration and Merging}: Input processing performs intelligent merging of multi-source data using store and date keys, handling potential mismatches and ensuring temporal alignment across all data sources.

\begin{lstlisting}[style=bashstyle, caption={Input Data Validation and Processing}]
# Validate required files
if not train_file or not features_file or not stores_file:
    st.error("All three CSV files are required")
    return

# Process input pipeline
df = load_and_merge_data(train_file, features_file, stores_file)
df = clean_data(df)
\end{lstlisting}

\textbf{Time Series Preparation}: The input pipeline aggregates data to weekly granularity, creates temporal indices, and applies differencing transformations required for stationarity. This includes handling irregular time intervals and missing observations through interpolation techniques.

\textbf{Feature Extraction}: Input processing extracts relevant features including holiday indicators, economic variables, and store characteristics, creating a comprehensive feature matrix that supports both univariate and multivariate forecasting approaches.

\textbf{Data Quality Assessment}: The pipeline implements robust quality checks including outlier detection, missing value analysis, and consistency verification across multiple time series to ensure high-quality input for model training.

\section{Training Process}

The training process implements systematic model development following machine learning best practices adapted for time series forecasting applications.

\textbf{Train-Test Split Strategy}: The system employs temporal splitting with 70\% data allocated for training and 30\% for testing, ensuring realistic evaluation that respects temporal ordering requirements.

\begin{lstlisting}[style=bashstyle, caption={Training Data Split and Model Fitting}]
# Configure train/test split
train_size = int(CONFIG['TRAIN_TEST_SPLIT'] * len(df_week_diff))
train_data_diff = df_week_diff[:train_size]
test_data_diff = df_week_diff[train_size:]

# Train selected model with hyperparameters
if model_type == "Auto ARIMA":
    model = train_auto_arima(train_data_diff, hyperparams)
else:
    model = train_exponential_smoothing(train_data_diff, hyperparams)
\end{lstlisting}

\textbf{Model Training Execution}: The training process supports both ARIMA and Exponential Smoothing algorithms with automated hyperparameter optimization, comprehensive error handling, and progress monitoring for user feedback.

\textbf{Cross-Validation Implementation}: While traditional k-fold cross-validation is inappropriate for time series data, the system implements time series-specific validation including walk-forward analysis and rolling window evaluation to ensure robust performance assessment.

\textbf{Model Persistence}: Trained models are serialized using both joblib and pickle methods to ensure cross-platform compatibility and reliable model storage for deployment applications.

\textbf{Training Monitoring}: The process includes comprehensive logging and progress monitoring, providing users with real-time feedback on training status, parameter optimization progress, and convergence criteria.

\section{Interpretation of Results}

The Walmart forecasting system achieved exceptional performance with a normalized WMAE of 3.58\%, placing it in the "Excellent" category for business forecasting applications. This performance demonstrates the effectiveness of the data mining approach for retail sales prediction.

\textbf{Performance Metrics Analysis}: The 3.58\% normalized WMAE indicates that forecasting errors average less than 4\% of actual sales values, providing high confidence for business planning applications. The absolute error of \$923.12 weekly provides concrete understanding of prediction accuracy in business terms.

\begin{lstlisting}[style=bashstyle, caption={Performance Evaluation and Interpretation}]
# Calculate detailed WMAE metrics
wmae_results = wmae_ts_detailed(test_data_diff, predictions)

# Business interpretation
interpretation, color_type = get_wmae_interpretation(wmae_results['normalized'])
# Result: "Excellent (less than 5% error)" with "success" status
\end{lstlisting}

\textbf{Business Impact Assessment}: The excellent performance enables reliable predictions for inventory management, with over 95\% accuracy supporting critical business decisions. Seasonal patterns are effectively captured, and holiday effects are properly modeled, providing actionable insights for operational planning.

\textbf{Model Validation}: Diagnostic plots confirm that the model successfully captures underlying time series patterns without systematic biases. Residual analysis indicates appropriate model specification with no evidence of significant autocorrelation in forecast errors.

\textbf{Comparative Performance}: The achieved performance significantly exceeds typical benchmarks for retail forecasting, demonstrating the effectiveness of the hybrid approach combining traditional statistical methods with modern optimization techniques.

\section{Output Generation and Visualization}

The system generates comprehensive output including forecasts, performance metrics, diagnostic visualizations, and downloadable models suitable for both technical analysis and business presentation.

\textbf{Forecast Generation}: The prediction system generates 4-week ahead forecasts with confidence intervals, providing point estimates and uncertainty quantification essential for business planning applications.

\textbf{Interactive Visualizations}: Output includes color-coded forecast charts with green indicators for positive growth and red indicators for declining trends, enabling intuitive interpretation of forecast results by non-technical stakeholders.

\begin{lstlisting}[style=bashstyle, caption={Output Generation and Export}]
# Generate forecast visualizations
fig = create_diagnostic_plots(train_data_diff, test_data_diff, predictions, model_type)

# Export model for deployment
joblib.dump(model, model_path)
st.download_button(label=f"Download {model_type} Model", data=model_file)
\end{lstlisting}

\textbf{Performance Reporting}: The system generates detailed performance reports including WMAE metrics, interpretation guides, and business impact assessments that translate technical results into actionable business insights.

\textbf{Export Capabilities}: Outputs are available in multiple formats including CSV and JSON downloads for integration with business systems, trained model files for deployment, and high-quality visualizations for presentation purposes.

\textbf{Real-time Dashboard}: The interactive interface provides real-time forecast generation with immediate visualization updates, enabling dynamic exploration of different scenarios and model configurations.

The comprehensive output generation ensures that data mining results are accessible, interpretable, and actionable for diverse stakeholder requirements ranging from technical model validation to strategic business planning.