%%%%%%%%%%%%%%%%%%%%%%%%
%
% $Autor: Hemanth Jadiswami Prabhakaran $
% $Datum: 2025-06-30 11:10:46Z $
% $Pfad: GitHub/BA25-01-Time-Series/report/Contents/en/domainKnowledge.tex $
% $Version: 1 $
%
% $Project: BA25-Time-Series $
%
%%%%%%%%%%%%%%%%%%%%%%%%

\chapter{Domain Knowledge}

\section{Domain Understanding}

Retail sales forecasting represents a critical domain within the broader field of time series analysis, particularly for large-scale retail operations such as Walmart, the world's largest retailer with over 11,500 stores globally \cite{Zhang:2021}. The domain of retail analytics encompasses multiple interconnected challenges including demand prediction, inventory optimization, supply chain management, and revenue forecasting across diverse product categories and geographic locations.

Within this domain, several key factors distinguish retail sales forecasting from other time series applications. First, retail sales exhibit multiple forms of seasonality, including weekly patterns (related to shopping behaviors), monthly cycles (corresponding to payroll cycles and shopping habits), and annual seasonality (driven by holidays, weather patterns, and cultural events) \cite{McElroy:2018}. Second, external economic factors such as unemployment rates, fuel prices, and consumer price indices significantly influence purchasing behavior, requiring multivariate modeling approaches \cite{Loyal:2017}. Third, special events and promotional activities create irregular patterns that deviate from normal seasonal trends, necessitating sophisticated anomaly detection and handling mechanisms.

The complexity of retail forecasting is further amplified by the hierarchical nature of the data, where predictions must be made at multiple aggregation levels including individual products, departments, stores, and regions. This hierarchical structure introduces challenges related to forecast reconciliation and the optimal allocation of inventory across different organizational levels \cite{Fildes:2019}.

\section{Problem Statement}

The primary problem addressed in this study involves developing accurate and reliable forecasting models for weekly sales data across multiple Walmart stores and departments. Specifically, the challenge encompasses predicting future sales values for over 4,400 unique time series, each representing the weekly sales of a specific department within a particular store over the period from February 2010 to November 2012.

The forecasting problem is characterized by several technical challenges. First, the high dimensionality of the dataset, with multiple stores and departments creating thousands of individual time series, requires scalable modeling approaches that can handle computational complexity while maintaining prediction accuracy. Second, the presence of irregular events such as holidays (Super Bowl, Labor Day, Thanksgiving, Christmas) creates non-stationary patterns that traditional time series models struggle to capture effectively \cite{Loyal:2017}.

Third, the integration of external economic variables introduces multicollinearity concerns and requires careful feature selection to avoid overfitting while capturing meaningful relationships between economic indicators and sales performance. Fourth, the weekly aggregation level, while computationally manageable, may obscure important daily patterns and intra-week dynamics that could provide valuable forecasting signals \cite{McElroy:2018}.

The practical business problem underlying this technical challenge involves enabling Walmart to optimize inventory management, staffing decisions, and promotional strategies through improved demand prediction. Accurate sales forecasts directly impact operational efficiency, customer satisfaction, and financial performance by reducing stockouts, minimizing excess inventory, and enabling proactive resource allocation.

\section{Data Acquisition}

The dataset utilized in this study was acquired from Kaggle, a prominent platform for data science competitions and datasets \href{https://www.kaggle.com/datasets/yasserh/walmart-dataset/data}{Kaggle - Walmart Dataset}. This publicly available dataset represents a subset of Walmart's historical sales data, specifically designed for academic research and machine learning competitions focused on retail sales forecasting.

The data acquisition process involved downloading three primary CSV files that collectively provide comprehensive information about Walmart's sales operations across multiple dimensions. The dataset's public availability through Kaggle ensures reproducibility of research findings and enables comparative analysis with other forecasting studies using the same data source.

The choice of this particular dataset was motivated by several factors. First, its real-world origin from a major retail corporation provides authentic patterns and challenges representative of actual business forecasting scenarios. Second, the inclusion of both sales data and external economic variables enables exploration of multivariate forecasting approaches. Third, the dataset's use in competitive machine learning environments has established benchmark performance metrics that facilitate objective evaluation of different modeling approaches.

\section{Data Quantity Assessment}

The Walmart sales dataset demonstrates substantial scale and temporal coverage, making it well-suited for comprehensive time series analysis. The dataset encompasses **45 distinct Walmart stores** located across different regions of the United States, with each store containing **multiple departments** that generate individual sales time series.

\textbf{Temporal Coverage}: The dataset spans approximately \textbf{2 years and 9 months}, from February 5, 2010, to November 1, 2012, providing \textbf{143 weeks} of historical data. This temporal range captures multiple complete seasonal cycles, including two full calendar years plus partial years, ensuring sufficient data for identifying annual, quarterly, and weekly seasonal patterns.

\textbf{Time Series Volume}: The combination of 45 stores and varying numbers of departments per store results in \textbf{over 4,400 unique time series} for analysis. This high-dimensional dataset provides substantial sample size for training robust forecasting models while presenting computational challenges that require efficient algorithmic implementations.

\textbf{External Variables}: In addition to sales data, the dataset includes \textbf{weekly observations} of four key economic indicators: temperature, fuel prices, consumer price index (CPI), and unemployment rates. These external variables provide \textbf{572 additional data points} (143 weeks × 4 variables) that can enhance forecasting accuracy through multivariate modeling approaches.

\textbf{Data Density}: The dataset exhibits high completeness with minimal missing values, ensuring robust model training without extensive imputation requirements. Each time series contains consistent weekly observations, providing reliable temporal structure for forecasting applications.

This data quantity is considered adequate for time series forecasting applications, as it provides sufficient historical observations to identify seasonal patterns, estimate model parameters with statistical confidence, and validate forecasting performance through out-of-sample testing \cite{Pao:2017}.

\section{Data Quality Evaluation}

\textbf{Completeness}: The Walmart dataset demonstrates high data completeness with minimal missing values across all three primary data files (train.csv, features.csv, stores.csv). The sales data (train.csv) contains complete weekly observations for all store-department combinations, ensuring consistent temporal coverage without gaps that could compromise forecasting model performance.

\textbf{Consistency}: Data consistency is maintained through standardized formatting and consistent variable definitions across all files. Store identifiers, date formats, and variable scales remain uniform throughout the dataset. The weekly aggregation level provides consistent temporal granularity that facilitates time series analysis without irregular intervals.

\textbf{Accuracy}: While direct validation of data accuracy against external sources is not feasible due to proprietary nature of Walmart's internal data, the dataset's origin from Kaggle competitions and its widespread use in academic research suggests reasonable accuracy levels. The presence of realistic sales patterns, including seasonal variations and holiday effects, supports the dataset's authenticity.

\textbf{Temporal Integrity}: The dataset maintains proper temporal ordering with consistent weekly intervals. Date stamps follow standardized format (YYYY-MM-DD), and no temporal gaps or irregularities were identified during preliminary analysis. This temporal integrity is crucial for time series modeling approaches that rely on sequential data structure.

\textbf{Variable Reliability}: Economic variables (temperature, fuel prices, CPI, unemployment) demonstrate reasonable ranges and variability consistent with expected economic patterns during the 2010-2012 period. Cross-validation of these variables against publicly available economic data could further enhance quality assessment, though such validation is beyond the scope of this analysis.

\textbf{Limitations}: The dataset's limitation to 45 stores represents only a small fraction of Walmart's total store network, potentially limiting generalizability. Additionally, the weekly aggregation level may obscure important daily patterns that could provide valuable forecasting signals \cite{McElroy:2018}.

\section{Data Relevance Analysis}

The Walmart sales dataset demonstrates high relevance for retail forecasting research and practical business applications. \textbf{Sales Variables}: The core dependent variable (Weekly\_Sales) directly addresses the primary forecasting objective, providing the target values necessary for supervised learning approaches. The sales data represents actual transaction volumes, making predictions directly applicable to business decision-making processes.

\textbf{Store Characteristics}: Store-level information including store type and size provides contextual variables that explain performance variations across locations. These characteristics enable stratified analysis and support development of store-specific forecasting models that account for operational differences.

\textbf{Economic Indicators}: The inclusion of four key economic variables (temperature, fuel prices, CPI, unemployment) enhances model relevance by capturing external factors that influence consumer behavior. These variables represent well-established economic drivers of retail demand and are commonly used in econometric forecasting models \cite{Zhang:2021}.

\textbf{Holiday Indicators}: The binary holiday flags (IsHoliday) provide essential information for modeling irregular patterns associated with major shopping events. These indicators are particularly relevant for retail forecasting as holidays significantly impact purchasing behavior and require special treatment in forecasting models \cite{Loyal:2017}.

\textbf{Temporal Scope}: The 2010-2012 timeframe covers a period of economic recovery following the 2008 financial crisis, providing relevant patterns for understanding retail performance during varying economic conditions. This temporal relevance enhances the dataset's applicability to similar economic environments.

\textbf{Business Applicability}: The dataset's structure directly supports practical business applications including inventory planning, staffing optimization, and promotional strategy development. The weekly granularity aligns with typical retail planning cycles, making forecasts immediately actionable for operational decision-making.

\section{Outlier Detection and Analysis}

**Identification Methodology**: Outlier detection in the Walmart sales dataset requires sophisticated approaches that distinguish between legitimate extreme values (such as holiday sales spikes) and data quality issues. Statistical methods including the Interquartile Range (IQR) method, Z-score analysis, and Isolation Forest algorithms are employed to identify potential outliers while preserving meaningful business events.

**Holiday-Related Outliers**: A significant portion of identified outliers corresponds to major holiday periods including Black Friday, Christmas, and Thanksgiving weeks. These outliers represent legitimate business phenomena rather than data quality issues. For example, Black Friday sales often exceed normal weekly volumes by 200-400%, creating extreme values that are essential to preserve for accurate holiday forecasting \cite{Loyal:2017}.

**Economic Event Outliers**: Certain outliers correlate with significant economic events during the 2010-2012 period, including fuel price spikes and unemployment rate changes. These outliers provide valuable information about retail sensitivity to economic conditions and should be retained for comprehensive model training.

**Store-Specific Outliers**: Some outliers are attributable to store-specific events such as grand openings, renovations, or temporary closures. These outliers require careful evaluation to determine whether they represent recurring patterns or one-time events that may not be predictive of future performance.

**Department-Level Variations**: Certain departments exhibit higher outlier frequencies due to their inherent sales volatility. Seasonal departments (lawn and garden, sporting goods) naturally display more extreme variations than stable categories (grocery, pharmacy), requiring department-specific outlier handling strategies.

**Treatment Strategies**: Rather than automatic outlier removal, the analysis employs domain-aware outlier treatment that preserves business-relevant extreme values while identifying potential data quality issues. This approach includes flagging outliers for further investigation, applying robust forecasting methods that handle extreme values, and developing separate models for high-volatility periods.

**Impact Assessment**: Outlier analysis reveals that approximately 3-5\% of observations exceed typical statistical thresholds, with the majority representing legitimate business events. Removal of all statistical outliers would eliminate crucial information about holiday performance and economic sensitivity, potentially degrading forecasting accuracy for the most commercially important periods.

\section{Anomaly Detection and Management}

\textbf{Anomaly Classification}: Anomalies in the Walmart sales dataset are classified into three primary categories: \textbf{seasonal anomalies} (deviations from expected seasonal patterns), \textbf{contextual anomalies} (unusual values given specific circumstances), and \textbf{collective anomalies} (patterns that appear normal individually but are anomalous as a group).

\textbf{Seasonal Anomaly Detection}: Advanced anomaly detection employs seasonal decomposition methods to identify deviations from expected seasonal trends. The STL (Seasonal and Trend decomposition using Loess) algorithm separates time series into trend, seasonal, and residual components, enabling identification of anomalies in the residual series that cannot be explained by normal seasonal patterns \cite{McElroy:2018}.

\textbf{Economic Context Anomalies}: Contextual anomaly detection incorporates economic variables to identify sales patterns that are unusual given prevailing economic conditions. For example, high sales during periods of elevated unemployment or fuel prices may indicate anomalous behavior requiring special attention in forecasting models.

\textbf{Store-Department Anomalies}: Collective anomaly detection identifies patterns where individual store-department combinations exhibit unusual behavior relative to their peer groups. This analysis helps identify systematic issues affecting specific locations or product categories that may require targeted modeling approaches.

\textbf{Holiday Anomaly Patterns}: Special attention is given to holiday periods, where anomalies may indicate successful promotional campaigns, supply chain disruptions, or changing consumer preferences. Holiday anomalies are preserved and flagged for enhanced modeling rather than removal, as they provide crucial information for future holiday forecasting.

\textbf{Temporal Anomaly Clustering}: Anomalies are analyzed for temporal clustering to identify periods of systematic unusual behavior. Extended periods of anomalous performance may indicate structural changes in consumer behavior, competitive pressures, or operational modifications that affect long-term forecasting assumptions.

\textbf{Impact on Forecasting Models}: Anomaly management strategies vary by model type. Traditional statistical models (ARIMA) may be more sensitive to anomalies and require robust parameter estimation techniques, while machine learning approaches may naturally accommodate anomalous patterns through training on diverse examples.

\textbf{Automated Anomaly Handling}: The forecasting system incorporates automated anomaly detection pipelines that flag unusual patterns for human review while maintaining operational efficiency. This approach balances the need for domain expertise in anomaly interpretation with the scalability requirements of processing over 4,400 time series.

\textbf{Documentation and Tracking}: All identified anomalies are documented with contextual information including economic conditions, holiday proximity, and potential business explanations. This documentation supports model validation, performance analysis, and continuous improvement of anomaly detection capabilities.

\textbf{Validation Framework}: Anomaly detection accuracy is validated through business expert review and retrospective analysis of flagged events. This validation ensures that the anomaly detection system effectively identifies genuine unusual patterns while minimizing false positives that could degrade forecasting performance.

\section{Data Source Citation}

This study utilizes the Walmart Sales Dataset available through Kaggle, a comprehensive collection of historical retail sales data spanning multiple stores and departments. The dataset provides weekly sales figures from 45 Walmart stores across different regions of the United States, covering the period from February 2010 to November 2012.

\textbf{Primary Data Source:}
\begin{itemize}
	\item \textbf{Title:} Walmart Dataset
	\item \textbf{Platform:} Kaggle
	\item \textbf{URL:} \url{https://www.kaggle.com/datasets/yasserh/walmart-dataset/data}
	\item \textbf{Accessed:} 2025
	\item \textbf{License:} Public Domain
\end{itemize}

The dataset consists of three primary files: train.csv (containing weekly sales data with store, department, date, and holiday indicators), features.csv (providing economic variables including temperature, fuel prices, CPI, and unemployment rates), and stores.csv (containing store characteristics such as type and size). This comprehensive data collection enables multivariate time series analysis incorporating both internal sales patterns and external economic factors.

The public availability of this dataset through Kaggle ensures reproducibility of research findings and facilitates comparative analysis with other forecasting studies. The dataset's widespread use in academic research and machine learning competitions has established it as a standard benchmark for retail sales forecasting applications, providing validated ground truth for model evaluation and performance assessment.