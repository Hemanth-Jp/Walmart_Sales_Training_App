%%%
%
% $Autor: Ayush Plawat $
% $Datum: 2021-05-14 $
% $Pfad: GitLab/CornerBlending $
% $Dateiname: DomainProblem
% $Version: 4620 $
%
% !TeX spellcheck = de_DE/GB
%
%%%


\chapter{From Development to Deployment}

\section{Introduction}
Moving a forecasting model from an exploratory notebook to a production-grade artefact demands disciplined engineering.  
For the Walmart Sales Forecasting project this means  
(i) codifying every assumption about the \emph{data contract};  
(ii) freezing the \emph{software bill of materials} so that any team member can rebuild the environment byte-for-byte;  
(iii) enforcing a \emph{deterministic file hierarchy} that doubles as living documentation; and  
(iv) treating every trained model as an immutable artefact with cryptographic provenance metadata.  
The subsections below expand each pillar in detail, following the project manual\cite{Box:2016}\cite{Montgomery:2008}\cite{Guyon:2003}.

%--------------------------------------------------------------------
\section{Data Structure} % 10 marks
A robust data structure underpins reproducible science and reliable operations.

\begin{itemize}
	\item \textbf{Canonical schema.}  All source files (\verb|train.csv|, \verb|features.csv|, \verb|stores.csv|) load into a long-format fact table with exactly five index columns\\
	\verb|Store|, \verb|Dept|, \verb|Date|, \verb|Weekly_Sales|, \verb|IsHoliday|.  
	Down-stream code asserts both the column set and the data-types before continuing.
	\item \textbf{Strict typing.}  \verb|Store| and \verb|Dept| are \verb|int16|; dates use \verb|datetime64[ns]|; monetary values are \verb|float32|, saving \mbox{38\,\%} RAM versus 64-bit floats.
	\item \textbf{Partitioned storage.}  Raw CSVs are written once to \verb|data/raw/YYYY/| so Git LFS can delta-compress while keeping history.  
	ETL writes Parquet to \verb|data/processed/| and tags each folder with a SHA-256 of the pipeline code (\verb|pipeline_sha|).
	\item \textbf{Audit columns.}  Every processed row carries \verb|ingested_at| (UTC) and \verb|pipeline_sha|, enabling point-in-time reproduction without polluting statistical features.
	\item \textbf{Referential integrity.}  Foreign-key checks between \verb|stores.csv| and the fact table run at job start; any mismatch aborts the run.
\end{itemize}

%--------------------------------------------------------------------
\section{Tools} % 20 marks
All tools are pinned in \texttt{requirements.txt} and therefore identical in CI, notebooks and production.

\begin{table}[h]
	\centering\small
	\begin{tabular}{|l|p{7.6cm}|}
		\hline
		Python 3.12.0 & single interpreter; clearer trace-backs; about 5\,\% faster \\ \hline
		pandas 2.2.2 / NumPy 1.26.4 & memory-optimised ETL, rolling windows, typed dtypes \\ \hline
		statsmodels 0.14.2 & ETS back-end, Ljung--Box, Jarque--Bera diagnostics \\ \hline
		pmdarima 2.0.4 & auto-ARIMA grid search, scikit-learn style API \\ \hline
		joblib 1.4.2 & compressed pickle, \verb|mmap_mode="r"| in prod \\ \hline
		Streamlit 1.32.0 & single GUI code-base for Training~App \& Prediction~App \\ \hline
		pytest 8.2 + coverage & unit / integration tests in GitHub Actions \\ \hline
		git 2.43 + hooks & blocks commits if tests fail or dependencies drift \\ \hline
	\end{tabular}
\end{table}

Quick start (cross-platform):

\begin{lstlisting}[language=bash,basicstyle=\ttfamily\small]
	python3.12 -m venv venv
	source venv/bin/activate     # Windows: venv\Scripts\activate
	pip install -r requirements.txt
\end{lstlisting}

%--------------------------------------------------------------------
\section{File Structure for Model Exchangements} % 10 marks
Verified ASCII tree:

\begin{lstlisting}[basicstyle=\ttfamily\footnotesize]
	walmart_forecasting/
	|- data/
	|  |- raw/YYYY/                  # immutable Kaggle CSVs
	|  |- processed/                 # Parquet after ETL
	|- models/
	|  |- arima/                     # one .pkl per (store, dept)
	|  |- registry/                  # CI-promoted, read-only
	|- WalmartSalesTrainingApp/
	|  |- train.py
	|  |- models/                    # staging before CI
	|- WalmartSalesPredictionApp/
	|  |- walmartSalesPredictionApp.py
	|  |- models/
	|     |- default/                # shipped with repo
	|- src/                          # shared Python code
	|- config/                       # YAML configs
	|- notebooks/                    # EDA / R&D
	|- tests/                        # pytest suites
\end{lstlisting}

Model path:
\begin{enumerate}
	\item Training-App writes \verb|arima__store-01__dept-01__v1.0.0.pkl| to its local \texttt{models/}.
	\item CI validates\,(tests+WMAE\,$\le$\,5\,\%) then copies the file and its JSON meta to \texttt{models/registry/}.
	\item Prediction-App lets the user upload the newest registry file into its private \texttt{models/default/}; checksum verification runs before the model is served.
\end{enumerate}

%--------------------------------------------------------------------
\section{Description of the Filetypes} % 25 marks

The deployment pipeline touches eight filetypes; the table below explains each in operational depth.

\begin{table}[H]
	\centering\footnotesize
	\begin{tabular}{|l|p{3.5cm}|p{8.4cm}|}
		\hline
		\textbf{File} & \textbf{Where used} & \textbf{Practical details and best practice} \\ \hline\hline
		\texttt{.csv} & Raw ingestion + some exports &
		UTF-8, LF.  Loaded with an explicit dtype map to stop pandas guessing.  
		Timestamps parsed via \verb|parse_dates|; duplicate primary keys abort ETL.  
		Large (>500 MB) files streamed in 10 k-row chunks so peak RAM stays < 1 GB. \\ \hline
		\texttt{.parquet} & \verb|data/processed/| &
		Column-oriented; Snappy compression saves about 70 \%.  
		Predicate push-down makes store-level slices 10× faster than CSV.  
		Schema evolution is safe; new columns append without rewriting old partitions. \\ \hline
		\texttt{.pkl} & Model artefacts (\texttt{models/}) &
		Written by \verb|joblib.dump(model, compress=3)|.  
		Name pattern \texttt{arima\_\_store-<nn>\_\_dept-<nn>\_\_vX.Y.Z.pkl} enforces semantic versioning.  
		Loaded with \verb|mmap_mode="r"| so multiple requests share RAM in Streamlit Cloud. \\ \hline
		\texttt{.json} & Side-car to every \texttt{.pkl} &
		Contains order, seasonal order, AIC, SHA-256 of training data, Git SHA and timestamp.  
		Language-agnostic, so ops can inspect provenance without Python.  
		Used at load-time for drift detection; mismatch aborts serving. \\ \hline
		\texttt{.yaml} & \texttt{config/} &
		Human-readable configs parsed by \verb|ruamel.yaml| with JSON-Schema validation.  
		Layered files allow environment overrides (\textit{dev}, \textit{stage}, \textit{prod}).  
		Editing a YAML triggers CI to rebuild the lock-file, preventing "works-on-my-machine" bugs. \\ \hline
		\texttt{.ipynb} & \texttt{notebooks/} &
		Executed nightly with \verb|nbconvert --execute|; failures break the build, ensuring examples never rot.  
		Outputs stripped before commit so diffs remain readable.  First cell prints current Git SHA for traceability. \\ \hline
		\texttt{.py} & \texttt{src/}, apps, CLI &
		PEP 8 + mypy-checked; public functions carry doc-strings with I/O contracts.  
		Imported by both Streamlit apps, guaranteeing single-source-of-truth logic. \\ \hline
		\texttt{.log} & \texttt{logs/} (rotating JSON lines) &
		Training: epoch stats, WMAE, CPU\,\%, RAM\,\%.  
		Prediction: request latency, 95\textsuperscript{th} percentile throughput.  
		Ingested by Grafana via Loki for real-time dashboards. \\ \hline
	\end{tabular}
	\caption{Filetypes and their role in the Walmart pipeline}
	\label{tab:filetypes}
\end{table}

Why multiple formats? CSV for universal exchange; Parquet for speed and compression; Pickle for binary weights; JSON for human-readable provenance; YAML for operator-friendly config; logs in JSON-lines for DevOps ingestion.

%--------------------------------------------------------------------
\section{Saving Models} % 5 marks

\begin{lstlisting}[language=Python,basicstyle=\ttfamily\small]
	from pathlib import Path
	import joblib, json, hashlib, pandas as pd, tempfile, shutil
	
	def save_model(model, store, dept, y_train):
	tag = f"store-{store:02d}__dept-{dept:02d}"
	out_dir = Path("WalmartSalesTrainingApp/models")
	out_dir.mkdir(parents=True, exist_ok=True)
	
	tmp = Path(tempfile.mkdtemp())
	pkl_tmp  = tmp / f"arima__{tag}__v1.0.0.pkl"
	meta_tmp = pkl_tmp.with_suffix(".json")
	
	joblib.dump(model, pkl_tmp, compress=3)
	
	sha = hashlib.sha256(
	pd.util.hash_pandas_object(y_train, index=True).values
	).hexdigest()
	json.dump({"store":store,"dept":dept,
		"order":model.order,"aic":model.aic(),
		"data_sha256":sha},
	open(meta_tmp, "w"), indent=2)
	
	shutil.move(pkl_tmp,  out_dir / pkl_tmp.name)
	shutil.move(meta_tmp, out_dir / meta_tmp.name)
\end{lstlisting}

%--------------------------------------------------------------------
\section{Loading Models} % 5 marks
\begin{lstlisting}[language=Python,basicstyle=\ttfamily\small]
	import joblib, json, hashlib, pandas as pd
	from pathlib import Path
	
	def secure_load(pkl_path: Path, y_input: pd.Series):
	meta = json.loads(pkl_path.with_suffix(".json").read_text())
	cur_sha = hashlib.sha256(
	pd.util.hash_pandas_object(y_input, index=True).values
	).hexdigest()
	if cur_sha != meta["data_sha256"]:
	raise ValueError("Feature drift detected -- abort")
	return joblib.load(pkl_path, mmap_mode="r")
\end{lstlisting}

%--------------------------------------------------------------------
\section{Development--Deployment Workflow}

\begin{center}
	\begin{tikzpicture}[
		node distance=1.1cm and 2.1cm,
		every node/.style={draw, rounded corners, minimum height=0.8cm,
			font=\footnotesize, align=center},
		process/.style={fill=orange!20},
		storage/.style={cylinder, shape aspect=0.3, fill=blue!15},
		arrow/.style={thick,->,>=stealth}]
		\node (raw)   [storage] {Raw\\CSV};
		\node (etl)   [process, below=of raw] {ETL\\+ QA};
		\node (feat)  [process, below=of etl]  {Feature\\Build};
		\node (train) [process, below=of feat] {AutoARIMA\\Train};
		\node (save)  [process, below=of train] {Save\\PKL+JSON};
		\node (reg)   [storage, below=of save] {Registry\\(CI)};
		\node (app)   [process, right=4cm of feat] {Prediction\\Streamlit};
		\node (api)   [process, below=of app] {REST\\Endpoint};
		
		\draw[arrow] (raw)  -- (etl);
		\draw[arrow] (etl)  -- (feat);
		\draw[arrow] (feat) -- (train);
		\draw[arrow] (train)-- (save);
		\draw[arrow] (save) -- (reg);
		\draw[arrow] (reg)  -- (app) node[midway,above]{\tiny upload / CI};
		\draw[arrow] (app)  -- (api);
	\end{tikzpicture}
\end{center}

%--------------------------------------------------------------------
\section{Conclusion}
A strict data contract, pinned tool-chain, ASCII-safe hierarchy and checksum-verified promotion guarantee that a model trained in the Streamlit Training-App is the exact artefact powering forecasts in the Prediction-App.  
This foundation supports future upgrades such as Docker images or MLflow registries while preserving the statistical rigour necessary for enterprise retail forecasting.
