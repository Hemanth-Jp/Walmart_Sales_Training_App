%%%%%%
%
% $Autor: Wings $
% $Datum: 2020-01-18 11:15:45Z $
% $Pfad: WuSt/Skript/Produktspezifikation/powerpoint/ImageProcessing.tex $
% $Version: 4620 $
%
%%%%%%

\definecolor{MapleColor}{rgb}{1,0.0,0.}
\definecolor{PythonColor}{rgb}{0,0.5,1.}
\definecolor{ShellColor}{rgb}{1,0,0.5}
\definecolor{FileColor}{rgb}{0.5,0.5,1.}

\newcommand{\MapleCommand}[1]{\textcolor{MapleColor}{\texttt{#1}}}
\newcommand{\PYTHON}[1]{\textcolor{PythonColor}{\texttt{#1}}}
\newcommand{\SHELL}[1]{#1}%\textcolor{ShellColor}{\texttt{#1}}}
\newcommand{\FILE}[1]{\textcolor{FileColor}{\texttt{#1}}}


\newcommand{\QUELLE}{\textcolor{red}{hier Quelle finden}}

\newcolumntype{L}[1]{>{\raggedright\arraybackslash}p{#1}} % linksbündig mit Breitenangabe

\newcommand{\GRAFIK}{\textcolor{red}{Grafik einfügen}}

\newcommand{\DEF}[1]{\fcolorbox{blue}{blue!10}{\begin{minipage}{\textwidth}\textbf{Definition.}#1\end{minipage}}}

\newcommand{\BEISPIEL}[1]{\fcolorbox{blue}{blue!10}{\begin{minipage}{\textwidth}\textbf{Beispiel.}#1\end{minipage}}}

\newcommand{\SATZ}[1]{\fcolorbox{blue}{blue!10}{\begin{minipage}{\textwidth}\textbf{Satz.}#1\end{minipage}}}

\newcommand{\Bemerkung}[1]{\fcolorbox{blue}{blue!10}{\begin{minipage}{\textwidth}\textbf{Bemerkung.}#1\end{minipage}}}


\newcommand{\Po}{\mathbb{P}}
\newcommand{\R}{\mathbb{R}}

\DeclareMathOperator{\Atan2}{Atan2}
\DeclareMathOperator{\sign}{sign}

% Auswahl der Sprache
% 1.Argument ist der Pfad ohne "en" oder "de"
% 2.Argument ist der Dateiname
\newcommand{\InputLanguage}[2]{
    \ifdefined\isGerman
    \input{#1de/#2}
    \else
    \ifdefined\isEnglish
    \input{#1en/#2}
    \else
    \input{#1de/#2}
    \fi
    \fi
}

\newcommand{\TRANS}[2]{
    \ifdefined\isGerman
    #1%
    \else
    \ifdefined\isEnglish
    #2%
    \else
    #1%
    \fi
    \fi
}


% muss für Akronyme \ac statt see verwendet werden.
\newcommand{\Siehe}{
    \ifdefined\isGerman
    \emph{siehe}
    \else
    \ifdefined\isEnglish
    \emph{see}
    \else
    \emph{siehe}
    \fi
    \fi
}


%todo Die Kommandos sind für das Endprodukt zu entfernen. Die entsprechenden Stellen sind zu bearbeiten bzw. zu löschen
\newcommand{\Mynote}[1]{\marginnote{\textcolor{red}{WS:#1}}}
\newcommand{\Ausblenden}[1]{}
\newcommand{\ToDo}[1]{\textcolor{red}{\section{ToDo} #1}}
% Fehlerzähler
\setul{0.5ex}{0.3ex}
\setulcolor{red}
\newcounter{fehlernummer}
\setcounter{fehlernummer}{11}
\newcommand{\FEHLER}[1]{\ul{#1}\stepcounter{fehlernummer}\textsuperscript{\textcolor{red}{\arabic{fehlernummer}}}}
%\renewcommand{\FEHLER}[1]{\ul{#1}\stepcounter{fehlernummer}\marginnote{\textcolor{red}{\arabic{fehlernummer}}}} geht leider nicht
%\renewcommand{\FEHLER}[1]{\ul{#1}\stepcounter{fehlernummer}\footnote{\textcolor{red}{\arabic{fehlernummer}.Fehler}}} geht leider nicht

\graphicspath{{../Images}}



\newcommand\pythonstyle{
	\lstset{ 
		backgroundcolor=\color{white},   % choose the background color; you must add \usepackage{color} or \usepackage{xcolor}; should come as last argument
		basicstyle=\footnotesize,        % the size of the fonts that are used for the code
		breakatwhitespace=false,         % sets if automatic breaks should only happen at whitespace
		breaklines=true,                 % sets automatic line breaking
		captionpos=b,                    % sets the caption-position to bottom
		commentstyle=\color{mygreen},    % comment style
		deletekeywords={...},            % if you want to delete keywords from the given language
		escapeinside={\%*}{*)},          % if you want to add LaTeX within your code
		extendedchars=true,              % lets you use non-ASCII characters; for 8-bits encodings only, does not work with UTF-8
		firstnumber=1000,                % start line enumeration with line 1000
		frame=single,	                   % adds a frame around the code
		keepspaces=true,                 % keeps spaces in text, useful for keeping indentation of code (possibly needs columns=flexible)
		keywordstyle=\color{blue},       % keyword style
		language=Python,                 % the language of the code
		morekeywords={*,...},            % if you want to add more keywords to the set
		numbers=left,                    % where to put the line-numbers; possible values are (none, left, right)
		numbersep=5pt,                   % how far the line-numbers are from the code
		numberstyle=\tiny\color{mygray}, % the style that is used for the line-numbers
		rulecolor=\color{black},         % if not set, the frame-color may be changed on line-breaks within not-black text (e.g. comments (green here))
		showspaces=false,                % show spaces everywhere adding particular underscores; it overrides 'showstringspaces'
		showstringspaces=false,          % underline spaces within strings only
		showtabs=false,                  % show tabs within strings adding particular underscores
		stepnumber=2,                    % the step between two line-numbers. If it's 1, each line will be numbered
		stringstyle=\color{mymauve},     % string literal style
		tabsize=2,	                     % sets default tabsize to 2 spaces
		title=\lstname                   % show the filename of files included with \lstinputlisting; also try caption instead of title
	} 
}

\lstset{language=Python}

% Python environment
\lstnewenvironment{Python}[1][]
{
	\pythonstyle
	\lstset{#1}
}
{}

%Python for external files
\newcommand\pythonexternal[2][]{{
		\pythonstyle
		\lstinputlisting[#1]{#2}}}

\DeclareCaptionType{code}[Listing][\TRANS{Liste des Listings}{List of Listings}] 

