%%%%%%%%%%%%%%%%%%%%%%%%
%
% $Autor: Hemanth Jadiswami Prabhakaran $
% $Datum: 2025-06-25 13:26:44Z $
% $Pfad: GitHub/BA25-01-Time-Series/Presentations/WalmartSalesForecastingPresentations/slides/systemArchitecture.tex $
% $Version: 1 $
%
% $Project: BA25-Time-Series $
%
%%%%%%%%%%%%%%%%%%%%%%%%



\Mysection{System Architecture \& Applications}

\STANDARD{Dual-Application Architecture}
{
  \framesubtitle{Complete Forecasting Workflow}
  
  \begin{figure}[H]
    \centering
    \begin{tikzpicture}[
	node distance=0.8cm,
	app/.style={rectangle, rounded corners, fill=HSELhellblau!30, draw=HSELblau, very thick, minimum width=1.8cm, minimum height=4cm, text width=1.6cm, align=center},
	feature/.style={rectangle, rounded corners, fill=white, draw=HSELblau, minimum width=1.5cm, minimum height=0.3cm, font=\scriptsize, text width=1.4cm, align=center},
	data/.style={rectangle, rounded corners, fill=yellow!20, draw=orange, thick, minimum width=1.2cm, minimum height=0.8cm, text width=1.1cm, align=center, font=\scriptsize}
	]
	
	% Horizontal flow: CSV Data -> Training App -> Model File -> Prediction App -> Results
	\node[data] (csv_data) {CSV Data \\ train.csv \\ features.csv \\ stores.csv};
	
	\node[app, right=of csv_data] (training) {};
	\node[above=0.05cm of training.north, font=\bfseries\scriptsize, color=HSELblau] {Training Application};
	\node[feature, below=0.1cm of training.north] (upload) {Data Upload};
	\node[feature, below=0.05cm of upload] (model_sel) {Model Selection};
	\node[feature, below=0.05cm of model_sel] (hyperparam) {Hyperparameter \\ Tuning};
	\node[feature, below=0.05cm of hyperparam] (evaluation) {Performance \\ Evaluation};
	\node[feature, below=0.05cm of evaluation] (export) {Model Export};
	
	\node[data, right=of training] (model_file) {Model File \\ (.pkl)};
	
	\node[app, right=of model_file] (prediction) {};
	\node[above=0.05cm of prediction.north, font=\bfseries\scriptsize, color=HSELblau] {Prediction Application};
	\node[feature, below=0.1cm of prediction.north] (load) {Model Loading};
	\node[feature, below=0.05cm of load] (generate) {Forecast \\ Generation};
	\node[feature, below=0.05cm of generate] (visualize) {Interactive \\ Visualization};
	\node[feature, below=0.05cm of visualize] (download) {Results Export};
	
	\node[data, right=of prediction] (results) {Forecasts \\ CSV/JSON \\ Charts};
	
	% Horizontal arrows with 0.5cm length
	\draw[->, thick, HSELblau] (csv_data.east) -- ++(0.5cm,0) -- (training.west);
	\draw[->, thick, HSELblau] (training.east) -- ++(0.5cm,0) -- (model_file.west);
	\draw[->, thick, HSELblau] (model_file.east) -- ++(0.5cm,0) -- (prediction.west);
	\draw[->, thick, HSELblau] (prediction.east) -- ++(0.5cm,0) -- (results.west);
	
\end{tikzpicture}
  \end{figure}

    \vspace{2cm}


  \textbf{Training Application}
  \begin{itemize}
    \item Model development \& validation
    \item Hyperparameter tuning
    \item Performance evaluation
    \item Model export capabilities
  \end{itemize}

  \textbf{Prediction Application}
  \begin{itemize}
    \item Production forecasting
    \item Interactive visualizations
    \item Real-time results
    \item Multiple export formats
  \end{itemize}
}

\MYNOTE{
  The system is designed with separation of concerns - training and prediction are handled by different applications for better workflow management.
}

% First slide - Technology Stack Architecture
\STANDARD{Technology Stack Architecture}
{
	\framesubtitle{System Overview and Component Structure}
	\begin{figure}[H]
		\centering
		\begin{tikzpicture}[scale=0.8, transform shape,
	node distance=0.24cm,
	layer/.style={rectangle, rounded corners, fill=HSELhellblau!40, draw=HSELblau, thick, minimum width=3.2cm, minimum height=0.8cm, font=\bfseries},
	tech/.style={rectangle, rounded corners, fill=white, draw=HSELblau, minimum width=1.44cm, minimum height=0.48cm, font=\small}
	]
	% Technology stack layers (bottom to top)
	\node[layer] (python) {Python 3.12 Foundation};
	\node[layer, above=0.56cm of python] (data_layer) {Data Processing Layer};
	\node[tech, above left=0.14cm and -1.05cm of data_layer] (pandas) {Pandas};
	\node[tech, above right=0.14cm and -1.05cm of data_layer] (numpy) {NumPy};
	\node[layer, above=0.56cm of data_layer] (ml_layer) {Machine Learning Layer};
	\node[tech, above left=0.14cm and -1.05cm of ml_layer] (statsmodels) {Statsmodels};
	\node[tech, above right=0.14cm and -1.05cm of ml_layer] (pmdarima) {pmdarima};
	\node[layer, above=0.56cm of ml_layer] (viz_layer) {Visualization Layer};
	\node[tech, above left=0.14cm and -1.05cm of viz_layer] (plotly) {Plotly};
	\node[tech, above right=0.14cm and -1.05cm of viz_layer] (matplotlib) {Matplotlib};
	\node[layer, above=0.56cm of viz_layer] (web_layer) {Web Framework};
	\node[tech, above=0.14cm of web_layer] (streamlit) {Streamlit};
	
	% Individual vertical arrows from layers to tech nodes
	\draw[->, HSELblau, thick] (data_layer.north) -- (pandas.south);
	\draw[->, HSELblau, thick] (data_layer.north) -- (numpy.south);
	\draw[->, HSELblau, thick] (ml_layer.north) -- (statsmodels.south);
	\draw[->, HSELblau, thick] (ml_layer.north) -- (pmdarima.south);
	\draw[->, HSELblau, thick] (viz_layer.north) -- (plotly.south);
	\draw[->, HSELblau, thick] (viz_layer.north) -- (matplotlib.south);
	\draw[->, HSELblau, thick] (web_layer.north) -- (streamlit.south);
	
	% Model types on the side - wider and shorter
	\node[right=1.2cm of ml_layer, rectangle, rounded corners, fill=green!20, draw=green!60, thick, minimum width=3.7cm, minimum height=1.9cm] (models) {};
	\node[above=0.08cm of models.north, font=\bfseries] {Models};
	\node[below=0.16cm of models.north] (arima) {Auto ARIMA};
	\node[below=0.08cm of arima] (exp_smooth) {Exponential Smoothing};
	\node[below=0.08cm of exp_smooth] (holt_winters) {Holt-Winters};
\end{tikzpicture}
	\end{figure}
}

% Second slide - Technology Details
\STANDARD{Technology Stack Details}
{
	\framesubtitle{Core Technologies and Implementation}
	\begin{columns}[T]
		\begin{column}{0.5\textwidth}
			\textbf{Core Technologies}
			\begin{itemize}
				\item \textbf{Python 3.12}: Exact version requirement
				\item \textbf{Streamlit}: Web application framework
				\item \textbf{Plotly}: Interactive visualizations
				\item \textbf{Pandas/NumPy}: Data processing
			\end{itemize}
		\end{column}
		
		\begin{column}{0.5\textwidth}
			\textbf{Forecasting Models}
			\begin{itemize}
				\item \textbf{Auto ARIMA}: Automated parameter selection
				\item \textbf{Exponential Smoothing}: Holt-Winters method
				\item \textbf{Joblib}: Model serialization
				\item \textbf{Statsmodels/pmdarima}: Implementation
			\end{itemize}
		\end{column}
	\end{columns}
}
\MYNOTE{
  The technology stack was chosen for rapid development, ease of deployment, and strong time series analysis capabilities.
}

\STANDARD{Data Pipeline}
{
  \framesubtitle{From Raw Data to Forecasts}
  \begin{figure}[!h]
    \centering
   \begin{tikzpicture}[
	node distance=0.5cm,
	box/.style={rectangle, rounded corners, minimum width=2.5cm, minimum height=0.8cm, font=\footnotesize},
	input/.style={box, fill=yellow!30, draw=orange, thick},
	process/.style={box, fill=HSELhellblau!30, draw=HSELblau, thick},
	output/.style={box, fill=green!30, draw=green!60, thick},
	arrow/.style={->, thick, HSELblau}
	]
	
	% Vertical flow
	\node[input] (data) {CSV Data Files (train, features, stores)};
	\node[process, below=of data] (validate) {Data Validation  Preprocessing};
	\node[process, below=of validate] (model) {Model Training  or Loading};
	\node[process, below=of model] (forecast) {4-Week Forecast Generation};
	\node[output, below=of forecast] (results) {Interactive Results  Export};
	
	% Arrows with exactly 0.5cm length
	\draw[arrow] (data.south) -- (validate.north);
	\draw[arrow] (validate.south) -- (model.north);
	\draw[arrow] (model.south) -- (forecast.north);
	\draw[arrow] (forecast.south) -- (results.north);
	
	% Side labels
	\node[left=1cm of data, font=\small\bfseries, orange] {Input};
	\node[left=1cm of validate, font=\small\bfseries, HSELblau] {Processing};
	\node[left=1cm of forecast, font=\small\bfseries, HSELblau] {};
	\node[left=1cm of results, font=\small\bfseries, green!60] {Output};
	
\end{tikzpicture}
  \end{figure}
  
  \textbf{Input Processing}
  \begin{itemize}
    \item \textbf{train.csv}: Historical sales data with store, date, weekly sales
    \item \textbf{features.csv}: External factors (temperature, fuel price, CPI, unemployment)
    \item \textbf{stores.csv}: Store metadata (type, size)
  \end{itemize}

  \textbf{Output Generation}
  \begin{itemize}
    \item \textbf{4-week forecasts}: Week-over-week sales changes
    \item \textbf{Interactive charts}: Color-coded visualizations
    \item \textbf{Export formats}: CSV, JSON for further analysis
  \end{itemize}
}

\MYNOTE{
  The data pipeline handles the complete workflow from data validation through preprocessing to forecast generation and visualization.
}

\STANDARD{Deployment Options}
{
  \framesubtitle{Flexible Access Methods}
  
  \begin{figure}[!h]
    \centering
    \begin{tikzpicture}[
	node distance=2cm,
	deployment/.style={rectangle, rounded corners, fill=HSELhellblau!40, draw=HSELblau, very thick, minimum width=4cm, minimum height=3cm},
	feature/.style={rectangle, rounded corners, fill=white, draw=HSELblau, minimum width=3.5cm, minimum height=0.5cm, font=\small},
	vs/.style={circle, fill=HSELblau, text=white, font=\bfseries\Large, minimum size=1cm}
	]
	
	% Move everything up and adjust spacing
	% Cloud deployment
	\node[deployment] (cloud) at (0,1) {};
	\node[above=0.2cm of cloud.north, font=\bfseries\large, HSELblau] {Cloud Deployment};
	\node[feature, below=0.3cm of cloud.north] (browser) {Browser Access};
	\node[feature, below=0.1cm of browser] (no_install) {No Installation};
	\node[feature, below=0.1cm of no_install] (auto_update) {Automatic Updates};
	\node[feature, below=0.1cm of auto_update] (cross_platform) {Cross-Platform};
	
	% VS symbol
	\node[vs, right=.7cm of cloud] {VS};
	
	% Local deployment - moved closer to center
	\node[deployment, right=2.4cm of cloud] (local) {};
	\node[above=0.2cm of local.north, font=\bfseries\large, HSELblau] {Local Installation};
	\node[feature, below=0.3cm of local.north] (performance) {Full Performance};
	\node[feature, below=0.1cm of performance] (offline) {Offline Capability};
	\node[feature, below=0.1cm of offline] (privacy) {Data Privacy};
	\node[feature, below=0.1cm of privacy] (large_data) {Large Datasets};
	
	% Decision factors - smaller and positioned lower
	\node[below=1cm of cloud.south, rectangle, rounded corners, fill=yellow!20, draw=orange, thick, minimum width=4.5cm, minimum height=1cm, xshift=3.2cm] (decision) {};
	\node[above=0.1cm of decision.north, font=\bfseries\footnotesize] {Choose Based On:};
	\node[below=0.03cm of decision.north, font=\small] {Quick Start → Cloud};
	\node[below=0.35cm of decision.north, font=\small] {Production Use → Local};
	\node[below=0.65cm of decision.north, font=\small] {Sensitive Data → Local};
	
\end{tikzpicture}
  \end{figure}

  \begin{columns}[T]
    \begin{column}{0.5\textwidth}
      \textbf{Cloud Deployment}
      \begin{itemize}
        \item Immediate browser access
        \item No installation required
        \item Automatic updates
        \item Cross-platform compatibility
      \end{itemize}
    \end{column}
    
    \begin{column}{0.5\textwidth}
      \textbf{Local Installation}
      \begin{itemize}
        \item Full performance control
        \item Offline capability
        \item Large dataset support
        \item Data privacy assurance
      \end{itemize}
    \end{column}
  \end{columns}
}

