%%%%%%
%
% Modified header.tex for English presentation
% Based on original template
%
% !TeX encoding = utf8
% !TeX root = walmart_slides
%
%%%%%%

%Packages
\usepackage[utf8]{inputenc} %Für Umlaute, da BibLaTeX
\usepackage[english]{babel} % Changed from german to english
\usepackage{amsmath}
\usepackage{amsfonts}
\usepackage{amssymb}
\usepackage{colortbl}
\usepackage{cancel} %'\cancel{}', '\bcancel{}' und '\xcancel{}'

%%%%%%%%%2-Screen%%%%%%%%%%%%%%%%%%%%%%%%%%%%%%%%%%%%%%%%%%%%
\usepackage{pgfpages} 
%%%%Kommentiert für Beamer
%%%%Aktiv für Notes
%\setbeameroption{show notes on second screen=bottom}
%\setbeameroption{second mode text on second screen=bottom}
%%%%%%%%%%%%%%%%%%%%%%%%%%%%%%%%%%%%%%%%%%%%%%%%%%%%%%%%%%%%%

%Für Grafiken
\usepackage{tikz}
\usetikzlibrary{mindmap}
\usepackage{gnuplottex}
\usepackage{pgf}
\usepackage{colortbl} 
\usetikzlibrary{calc}
\usetikzlibrary{shapes,arrows} %Für Flowchart
\usetikzlibrary{shapes.geometric} %Für Flowchart
\usetikzlibrary{positioning} % For relative positioning
\usetikzlibrary{decorations.pathreplacing} % For braces
\usetikzlibrary{fit} % For fitting nodes
\usepackage{scalefnt}
\usetikzlibrary{decorations.markings} %Für => pfeile
\usetikzlibrary{calc,patterns,decorations.pathmorphing,decorations.markings}

%Für urls in Quellen
\usepackage{url}

\usepackage[natbib=true,style=alphabetic,backend=bibtex,useprefix=true]{biblatex}

%Formatierungen
\mode<presentation>
\setbeamertemplate{headline} 
{%
\begin{beamercolorbox}[rounded=true, center]{bgcolor}
\begin{columns}[T]
\begin{column}{9cm}
{\color{gray}\begin{tiny}Hochschule Emden/Leer\end{tiny}} \\ 
{\color{gray}\begin{tiny}Abteilung Maschinenbau\end{tiny}} \\ 
{\color{gray}\begin{tiny}MSR-Labor\end{tiny}} \\
\end{column}
\begin{column}{2cm}
\includegraphics[scale=0.25]{img/technik.jpg}
\end{column}
\end{columns}
\end{beamercolorbox}
 }
\insertsectionhead
\insertsubsectionhead
\usetheme{default}
\useinnertheme[shadow=true]{rounded}
\usebackgroundtemplate
{%
      \rule{0pt}{\paperheight}%
      \hspace*{\paperwidth}%
      \makebox[0pt][r]{\includegraphics[width=\paperwidth]{img/hintergrund2.png}}
 }

\definecolor{HSELhellblau}{RGB}{138,198,203}
\definecolor{HSELblau}{RGB}{0,59,95}
\usecolortheme[named=HSELblau]{structure}
 \newcommand{\topline}{%
  \tikz[remember picture,overlay] {%
    \draw[HSELhellblau] ([yshift=-0.9cm]current page.north west)
             -- ([yshift=-0.9cm,xshift=\paperwidth]current page.north west);}}
\setbeamertemplate{section}[numbered]

\newcommand{\STANDARD}[2]
{
  \mode<presentation>%
  {%
     \begin{frame}[allowframebreaks]{#1} #2 \end{frame}
  }%
  \mode<article>
  {
    \fcolorbox{AliceBlue}%{Bisque} %{BlanchedAlmond}
    {LightGrey} %{Beige}   %{AliceBlue}
    {
      \begin{minipage}{\textwidth}{\bf #1} #2  \end{minipage}
    }%
    \medskip
    \hrulefill
  }
}

\newcommand{\MYNOTE}[1]
{
  \mode<presentation>%
  {%
     %\note{#1}
     \only<article>{#1}
  }%
  \mode<article>
  {
    #1
  }
}

% ------------
% sectionframe
% ------------
\newcommand{\sectionframe}[1]%
{%
	\begin{frame}
		\Huge
		\begin{center}
			#1 
		\end{center}
	\end{frame}%
}

\newcommand{\Mysection}[1]%
{%
  \section{#1}%
  \sectionframe{#1}%
}

\usepackage{listings}

% Farben für Syntax-Highlighting
\definecolor{dkgreen}{rgb}{0,.6,0}
\definecolor{dkblue}{rgb}{0.655,0.113,.364}
\definecolor{dkyellow}{cmyk}{0,0,.8,.3}

\definecolor{parameterc}{rgb}{.4,0,.6}
\definecolor{typec}{rgb}{0,0.525,.702}
\definecolor{stringc}{rgb}{0,.5019,.5019}
\definecolor{keywordc}{rgb}{.6549, .1137, .3647}
\definecolor{commentc}{rgb}{.5882, .5960, .5882}
\definecolor{textc}{rgb}{.2,.2,.2}

\lstdefinestyle{all}{
	alsoletter={-},
	frame=single,
	numbers=none,
	numberstyle=\tiny\color{textc},
	basicstyle=\linespread{0.9}\ttfamily\footnotesize\color{textc},
	tabsize=4,
	showstringspaces=false,
	captionpos=t,
	rulecolor=\color{lightgray!40},
	keywordstyle=\color{keywordc},
	stringstyle=\color{stringc},
	commentstyle=\color{commentc},
	breaklines=true,
	escapechar="!",
	postbreak=\mbox{\textcolor{green}{$\hookrightarrow$}\space},
}

\definecolor{PythonColor}{rgb}{0,0.5,1.}
\newcommand{\PYTHON}[1]{\textcolor{PythonColor}{\texttt{#1}}}
\definecolor{PythonColorHighLite}{rgb}{0.5,0,1.}
\newcommand{\PYTHONHL}[1]{\textcolor{PythonColorHighLite}{\texttt{#1}}}
\definecolor{ShellColor}{rgb}{0,1,1.}
\newcommand{\SHELL}[1]{\textcolor{ShellColor}{\texttt{#1}}}
\definecolor{FileColor}{rgb}{1,0,1.}
\newcommand{\FILE}[1]{\textcolor{FileColor}{\texttt{#1}}}